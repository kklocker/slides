%\usebackgroundtemplate{%             declare it
%	\tikz[overlay,remember picture] \node[opacity=1, at=(current page.center)] {
%		\includegraphics[height=\paperheight]{img/bakgrunn.png}};}

\begin{frame}[noframenumbering, plain]
	\FrameText{@\href{https://www.twitter.com/Karlockert}{karlockert}}
%	\frametitle{Hvis du ønsker}
	
%	\begin{block}{\color{white}\textbf{\Large{Ehrenfests teorem }}}
%		\vspace{-10pt}\rule{\textwidth}{0.5pt}
%		\color{white}
%	Hvorfor kollapser ikke bølgefunksjonen til hverdagslige ting (som f.eks. en stol)? Korrespondanseprinsippet forteller oss at i grensen av store antall partikler og høye energier, må kvantemekanikk være ekvivalent med klassisk fysikk. 
%	Paul Ehrenfest fant en måte å relatere tidsutviklingen i forventningsverdien av kvantemekaniske operatorer til forventningsverdien av kraften som virker på systemet. Fra Newtons andre lov er dette forventet, men Ehrenfests teorem gir en matematisk trygghet. En generalisering av teoremet kan skrives på formen
%	\end{block}
%	{\large
%		
%		\begin{equation*} 
%			\dv{t}\ev{O} = 	-i\hbar\ev{\comm{O}{\mathcal{H}}} + \ev{\pdv{O}{t}}	
%		\end{equation*}
%	}
%
%	hvor $O$ er en operator som korresponderer til en observerbar størrelse og $\mathcal{H}$ er Hamiltonian til systemet.
\end{frame}