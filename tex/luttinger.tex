
%\usebackgroundtemplate{%             declare it
%	\tikz[overlay,remember picture] \node[opacity=1, at=(current page.center)] {
%		\includegraphics[height=\paperheight]{img/bakgrunn.png}};}

\begin{frame}[noframenumbering, plain]
    \begin{block}{\color{white}\textbf{\Large{
                    %
                    Bosonization of fermions in one dimension
                    %	
                }}}
        \vspace{-10pt}\rule{\textwidth}{0.5pt}
        \color{white}
        Fermions, as we know, are subject to the Pauli Exclusion principle.
        This is also true in one dimension, with the dramatic consequence that single-particle excitations are singular (no electrons can  ``move'' past another)!
        The field operator of a given fermion species can be written in terms of bosonic(!) operators $a_\mu^\dagger, a_\mu$ as
    \end{block}
    \vspace*{-13pt}
    \[ \psi_{\mu}(x) = F_{\mu} \frac{1}{\sqrt{L}}\mathrm{e}^{-i\frac{2\pi}{L}\hat{N}_{\mu}x}\mathrm{e}^{-i\phi_{\mu}^{\dagger}(x)}\mathrm{e}^{-i\phi_{\mu}(x)} \]
    with
    \[ \phi_{\mu}^{\dagger}(x) \equiv -i\sum_{q>0}\left (\frac{2\pi}{L|q|}\right )^{\frac12}\mathrm{e}^{-\xi\frac{|q|}{2}}\mathrm{e}^{-iqx}a_{q,\mu}^{\dagger}. \]

    \vspace*{-5pt}
    \begin{block}{}
        \color{white}
        Here, $F_\mu$ is a ``Klein-operator'', increasing the occupancy number $N
            _\mu$ of fermion species $\mu$ relative to the vacuum. Luttinger showed that in one dimension any interaction breaks the discontinuity at the Fermi-surface. This renders traditional Fermi-liquid theory inapplicable to these systems.
    \end{block}
\end{frame}