%\usebackgroundtemplate{%             declare it
	%	\tikz[overlay,remember picture] \node[opacity=1, at=(current page.center)] {
		%		\includegraphics[height=\paperheight]{img/bakgrunn.png}};}

\begin{frame}[noframenumbering, plain]
	\begin{block}{\color{white}\textbf{\Large{
					%
				    Liouville's theorem (1838)
					%	
		}}}
		\vspace{-10pt}\rule{\textwidth}{0.5pt}
		\color{white}
		
	Liouville's theorem is is a key insight in Hamiltonian mechanics, and a fundamental theorem of statistical physics. It states that under canonical transformations, an arbitrary volume of phase space is left invariant.

	\end{block}
	{\large
		
		\begin{equation*} 
			\pdv{\rho}{t} + \sum_k\left(\pdv{\rho}{q_k}\pdv{q_k}{t} + \pdv{\rho}{p_k}\pdv{p_k}{t}\right) = 0 
		\end{equation*}
	}
	
	\begin{block}{}
		\color{white}
		Here, $\rho$ is the phase space density, the number of state configurations inside a given phase space volume. $q_k$ and $p_k$ are canonical coordinates and momenta, respectively. 
		You may recognize the second term as the Poisson bracket $\{\rho, H\}$
		
	\end{block}
	
	
\end{frame}