\begin{frame}[noframenumbering, plain]
	%\frametitle{Hvis du ønsker}
	
	\begin{block}{\color{white}\textbf{\Large{Hubbard-Stratonovich-transformasjonen}}}
		\vspace{-10pt}\rule{\textwidth}{0.5pt}
		\color{white}
		HS-transformasjon, også kalt HS-``dekobling'' er en nyttig identitet dersom du har for mange vekselvirkende fermioner i systemet ditt og heller ønsker en teori av (forhåpentligvis frie) bosoner.
		
	\end{block}
	
	{\large
		\begin{equation*}
			\e^{-\frac{a}{2}\psi^2} = \frac{1}{\sqrt{2\pi a}} \int\limits_{-\infty}^\infty\dd{\varphi} \e^{-\left(\frac{\varphi^2}{2a} +i\varphi\psi\right)}
		\end{equation*}
	}
	
	En diskret versjon av HS-dekoblingen kan for eksempel brukes til å transformere den kvantemekaniske Hubbard-modellen i $d$ dimensjoner til en klassisk Ising-liknende\footnote{\color{white}Multi-spin-vekselvirkende, i motsetning til den klassiske Ising-spin, som kun tar med par-spin-vekselvirkninger} modell i $d+1$ dimensjoner!
	
%	\fullcite{HSdecoupling}
	%\printbibliography
	%\bibliography{bib}
\end{frame}