\begin{frame}[noframenumbering, plain]
	%\frametitle{Hvis du ønsker}
	
	\begin{block}{\color{white}\textbf{\Large{Hubbard-Stratonovich-decoupling}}}
		\vspace{-10pt}\rule{\textwidth}{0.5pt}
		\color{white}
		The HS-decoupling is a useful identity if you are troubled by too many interacting fermions in your system and would prefer a theory written in terms of (hopefully) noninteracting bosons. 
		The transformation is based on the identity
	\end{block}
	
	{\large
		\begin{equation*}
			\e^{-\frac{a}{2}\psi^2} = \frac{1}{\sqrt{2\pi a}} \int\limits_{-\infty}^\infty\dd{\varphi} \e^{-\left(\frac{\varphi^2}{2a} +i\varphi\psi\right)},
		\end{equation*}
	}
	where $\psi$ represent fermions through Gra\ss man variables, and $\varphi$ represent bosonic field(s).
	This transformation is used in all kinds of quantum field theories. A discrete version can be used to transform the \emph{quantum mechanical} Hubbard model in $d$ dimensions into a \emph{classical} Ising-like model in $d+1$ dimensions. 
	
	
	
	
%	En diskret versjon av HS-dekoblingen kan for eksempel brukes til å transformere den kvantemekaniske Hubbard-modellen i $d$ dimensjoner til en klassisk Ising-liknende\footnote{\color{white}Multi-spin-vekselvirkende, i motsetning til den klassiske Ising-spin, som kun tar med par-spin-vekselvirkninger} modell i $d+1$ dimensjoner!
	
%	\fullcite{HSdecoupling}
	%\printbibliography
	%\bibliography{bib}
\end{frame}